\documentclass[12pt]{article}
\usepackage{makeidx}
\usepackage{graphicx}
\usepackage{listings}
\usepackage{color}

\usepackage[utf8]{inputenc}
\usepackage{amssymb}
\usepackage{fixltx2e}
\usepackage[left=2cm,right=2cm,top=2cm,bottom=5cm]{geometry}
\textwidth 6in
\textheight 9in
\topmargin 0in
\headsep 0in
\oddsidemargin 0.5cm
\evensidemargin -0.5cm
\hyphenchar\font=-1

\definecolor{dkgreen}{rgb}{0,0.6,0}
\definecolor{gray}{rgb}{0.5,0.5,0.5}
\definecolor{mauve}{rgb}{0.58,0,0.82}

\lstset{frame=tb,
  language=Java,
  aboveskip=3mm,
  belowskip=3mm,
  showstringspaces=false,
  columns=flexible,
  basicstyle={\small\ttfamily},
  numbers=none,
  numberstyle=\tiny\color{gray},
  keywordstyle=\color{blue},
  commentstyle=\color{dkgreen},
  stringstyle=\color{mauve},
  breaklines=true,
  breakatwhitespace=true,
  tabsize=2,
  inputencoding=utf8,
  extendedchars=true,
  literate={á}{{\'a}}1 {è}{{\'e}}1 {ì}{{\'i}}1 {ò}{{\'o}}1 {ù}{{\'u}}1
}


\title{Sicurezza dell'informazione M}
\author{Lorenzo Angelini (N. matricola 0000749514)}
\date{}

\begin{document}

\maketitle

\section*{Descrizione del progetto}

Il progetto ha come scopo quello di creare un canale di comunicazione sicuro tra due applicativi Java (\textit{Client} e \textit{Server}) collegati ad Internet.
Gli applicativi utilizzano come canale di comunicazione non sicuro una \textit{Socket TCP} e implementano il protocollo di \textit{Diffie Hellman Anonimo} per accordarsi su una chiave di cifratura simmetrica che useranno poi per cifrare/decifrare i dati trasmessi con algoritmo \textit{AES} in modalità \textit{CBC}.
\par
In questo modo i due host sono in grado di comunicare in maniera riservata, senza che un intruso possa capire quali informazioni si stanno scambiando sul canale.
Infatti, pur potendo intercettare l'intera comunicazione tra i due hosts (che comprende le chiavi pubbliche di Diffie Hellman di entrambi ed il contenuto della comunicazione in forma cifrata), l'intruso non sarà in grado di decifrare nessuna informazione.
\par
Come esempio di utilizzo il \textit{Client}, una volta stabilito un canale riservato con il \textit{Server}, gli chiede di essere informato ogni 2 secondi sul suo carico di sistema fino ad un massimo di $N$ volte.

\section*{Implementazione del protocollo di Diffie-Hellman}

Di seguito sono descritti i passi grazie ai quali \textit{Client} e \textit{Server} riescono ad ottenere una chiave comune $K$:

\begin{enumerate}

\item Il \textit{Client} genera i parametri $p$ e $g$
$$
p: \textit{numero primo}
$$
$$
g: \textit{generatore del gruppo moltiplicativo degli interi modulo p}
$$

\item Il \textit{Client} genera un numero casuale a e calcola il valore di $A = g^a \bmod p$ (dove $mod$ indica l'operazione modulo, ovvero il resto della divisione intera).
\\
In questo modo ha creato una coppia di chiavi, una pubblica $PK$ ed una privata $SK$, definite come:
$$
SK = a
$$
$$
PK = (g,p,A)
$$

\item Il \textit{Client} invia la sua chiave pubblica DH al \textit{Server}

\item Il \textit{Server}, grazie ai parametri contenuti nella chiave pubblica DH del \textit{Client} può generare a sua volta una coppia di chiavi definite come:
$$
SK = b = \textit{numero casuale generato dal Server}
$$
$$
PK = B = g^b \bmod p
$$

\item Il \textit{Server} invia la sua chiave pubblica DH $PK$ al Client

\item Il \textit{Client} calcola la chiave a 128 bit $K$
$$
K = serverPK^a \bmod p = B^a \bmod p = g^{ab} \bmod p
$$

\item Il \textit{Server} calcola la chiave a 128 bit $K$
$$
K = clientPK^b \bmod p = A^b \bmod p = g^{ab} \bmod p
$$

\item Le chiavi calcolate $K$ sono identiche

\end{enumerate}

\noindent
A questo punto i due interlocutori sono entrambi in possesso della chiave segreta $K$ e possono cominciare ad usarla per cifrare le comunicazioni successive.
\par
Un attaccante può benissimo ascoltare tutto lo scambio, ma per calcolare i valori $a$ e $b$ avrebbe bisogno di risolvere l'operazione del logaritmo discreto, che è computazionalmente onerosa e richiede parecchio tempo, in quanto sub-esponenziale (sicuramente molto più del tempo di conversazione tra i 2 interlocutori).


\section*{Algoritmo di cifratura AES-128 CBC}

Per la cifratura/decifratura dei dati trasmessi viene utilizzato l'algoritmo \textit{AES} (Advanced Encryption Standard) impiegando la chiave da 128 bit concordata precedentemente.
\par
\textit{AES} è un algoritmo di cifratura a blocchi utilizzato come standard dal governo degli Stati Uniti d'America.
AES è stato scelto perchè è tra gli algoritmi di cifratura più sicuri, si pensi che \textit{NSA} ritiene che una chiave a 128 bit impiegata con algoritmo \textit{AES} sia adeguata per proteggere documenti classificati come \textit{SECRET}, per documenti classificati come \textit{TOP SECRET} si rende invece necessaria una chiave da 192 o 256 bit.

\par
\textit{AES} viene utilizzata in modalità \textit{CBC} (Cipher Block Chaining) per risolvere il problema del determinismo di cui è affetta invece la modalità \textit{ECB}: per uno stesso blocco di testo in chiaro viene sempre generato lo stesso testo cifrato.
Questa peculiarità potrebbe fornire ad un eventuale attaccante una vulnerabilità da studiare per cercare di scoprire la parola chiave utilizzata.
\par
\textit{CBC} invece, prima di cifrare, somma modulo 2 il blocco di testo in chiaro con il precedente blocco cifrato. Questo comporta che ogni blocco di testo cifrato generato dipenda in maniera diretta dai blocchi di testo cifrato precedente.
Il primo blocco di testo in chiaro verrà sommato modulo 2 con un vettore di inizializzazione che viene generato dal \textit{Client} e inviato in chiaro al \textit{Server}.

\section*{Codice sorgente}

\subsection*{Test}
\begin{lstlisting}
import diffiehellman.client.ClientThread;
import diffiehellman.server.ServerThread;


public class Test {

	public static void main(String[] args) {

		(new Thread(new ServerThread())).start();
		(new Thread(new ClientThread())).start();

	}
}

\end{lstlisting}

\subsection*{ClientThread}
\begin{lstlisting}
package diffiehellman.client;

import java.io.IOException;
import java.io.ObjectInputStream;
import java.io.ObjectOutputStream;
import java.net.Socket;
import java.security.AlgorithmParameterGenerator;
import java.security.AlgorithmParameters;
import java.security.KeyFactory;
import java.security.KeyPair;
import java.security.KeyPairGenerator;
import java.security.PublicKey;
import java.security.spec.X509EncodedKeySpec;

import javax.crypto.Cipher;
import javax.crypto.KeyAgreement;
import javax.crypto.SecretKey;
import javax.crypto.spec.DHParameterSpec;

public class ClientThread implements Runnable {

	private static int serverPort = 8412;
	private static String serverAddr = "localhost";

	public void run() {

		// Inizializzazione della Socket
		Socket server = null;
		try {
			server = new Socket(serverAddr, serverPort);
		} catch (IOException e) {
			System.out.println("Unable to reach server");
			e.printStackTrace();
			return;
		}

		// Crea stream di in/out
		ObjectOutputStream outSocket = null;
		ObjectInputStream inSocket = null;
		try {
			outSocket = new ObjectOutputStream(server.getOutputStream());
			inSocket = new ObjectInputStream(server.getInputStream());
		} catch (IOException e) {
			System.out.println("Unable to get socket streams");
			e.printStackTrace();
			return;
		}

		// Adesso la connessione è correttamente stabilita
		try {
			SecretKey key = AESDHKeyAgreement(outSocket, inSocket);

			String times = String.valueOf(4);
			sendAESCryptedString(times, key, outSocket);

			String cleartext;
			do {
				cleartext = receiveAESCryptedString(key, inSocket);
				System.out.println("Client: decrypted text: "+cleartext);
			} while(!cleartext.equals("stop"));

		} catch (Exception e) {
			System.out.println("Client: ERROR");
			e.printStackTrace();
		}
	}

	public SecretKey AESDHKeyAgreement(ObjectOutputStream outSocket, ObjectInputStream inSocket) throws Exception {
		/* Il Client genera i parametri g e p, operazione costosa
		 * p: numero primo
		 * g: generatore del gruppo moltiplicativo degli interi modulo p
		 */
		System.out.println("Creating Diffie-Hellman parameters (takes VERY long) ...");
		AlgorithmParameterGenerator paramGen = AlgorithmParameterGenerator.getInstance("DH");
		paramGen.init(1024);
		AlgorithmParameters params = paramGen.generateParameters();
		DHParameterSpec dhSkipParamSpec = (DHParameterSpec) params.getParameterSpec(DHParameterSpec.class);

		/* Il Client crea la sua coppia di chiavi DH utilizzando i parametri generati sopra
		 * SK = a: numero casuale
		 * PK = (g,p,A)
		 * A: (g^a) mod p
		 */
		System.out.println("Client: Generate DH keypair ...");
		KeyPairGenerator clientKpairGen = KeyPairGenerator.getInstance("DH");
		clientKpairGen.initialize(dhSkipParamSpec);
		KeyPair clientKpair = clientKpairGen.generateKeyPair();

		// Il Client codifica la sua chiave pubblica e la invia al Server
		byte[] clientPubKeyEnc = clientKpair.getPublic().getEncoded();
		outSocket.writeObject(clientPubKeyEnc);

		/*
		 * Il Client riceve la chiave pubblica DH del Server in forma codificata.
		 * Istanzia un oggetto PublicKey dai dati che ha ricevuto
		 */
		byte[] serverPubKeyEnc = (byte[]) inSocket.readObject();
		KeyFactory clientKeyFac = KeyFactory.getInstance("DH");
		X509EncodedKeySpec x509KeySpec = new X509EncodedKeySpec(serverPubKeyEnc);
		PublicKey serverPubKey = clientKeyFac.generatePublic(x509KeySpec);

		/* Il Client crea ed inizializza un oggetto KeyAgreement di DH che,
		 * grazie alla SK del Client e alla PK del Server,
		 * può calcolare la chiave concordata K
		 * K = (serverPK^a) mod p = (B^a) mod p = (g^ab) mod p
		 */
		System.out.println("Client: Initialization ...");
		KeyAgreement clientKeyAgree = KeyAgreement.getInstance("DH");
		clientKeyAgree.init(clientKpair.getPrivate());
		System.out.println("Client: calculating agreed KEY ...");
		clientKeyAgree.doPhase(serverPubKey, true);

		/*
		 * A questo punto sia il Client che il Server hanno completato il protocollo di DH.
		 * Hanno ottenuto entrambi la chiave concordata K
		 */
		return clientKeyAgree.generateSecret("AES");
	}

	private void sendAESCryptedString(String cleartext, SecretKey key, ObjectOutputStream outSocket) throws Exception {
		Cipher cipher = Cipher.getInstance("AES/CBC/PKCS5Padding");
		cipher.init(Cipher.ENCRYPT_MODE, key);
		byte[] encodedParams = cipher.getParameters().getEncoded();
		outSocket.writeObject(encodedParams);
		byte[] ciphertext = cipher.doFinal(cleartext.getBytes());
		outSocket.writeObject(ciphertext);
	}

	private String receiveAESCryptedString(SecretKey key, ObjectInputStream inSocket) throws Exception {
		byte[] encodedParams = (byte[]) inSocket.readObject();
		AlgorithmParameters params = AlgorithmParameters.getInstance("AES");
		params.init(encodedParams);
		Cipher cipher = Cipher.getInstance("AES/CBC/PKCS5Padding");
		cipher.init(Cipher.DECRYPT_MODE, key, params);
		byte[] ciphertext = (byte[]) inSocket.readObject();
		byte[] recovered = cipher.doFinal(ciphertext);
		return new String(recovered);
	}
}
\end{lstlisting}

\subsection*{ServerThread}
\begin{lstlisting}
package diffiehellman.server;

import java.io.IOException;
import java.io.ObjectInputStream;
import java.io.ObjectOutputStream;
import java.lang.management.ManagementFactory;
import java.net.ServerSocket;
import java.net.Socket;
import java.security.AlgorithmParameters;
import java.security.KeyFactory;
import java.security.KeyPair;
import java.security.KeyPairGenerator;
import java.security.PublicKey;
import java.security.spec.X509EncodedKeySpec;

import javax.crypto.Cipher;
import javax.crypto.KeyAgreement;
import javax.crypto.SecretKey;
import javax.crypto.interfaces.DHPublicKey;
import javax.crypto.spec.DHParameterSpec;

public class ServerThread implements Runnable {

	private static int serverPort = 8412;

	public void run() {
		// Socket in ascolto su localhost
		ServerSocket server = null;
		try {
			server = new ServerSocket(serverPort);
			System.out.println("Server listening on "+serverPort);
		} catch (IOException e) {
			System.out.println("Error while starting server");
			e.printStackTrace();
			return;
		}

		// Loop infinito di accettazione richieste di connessione
		while(true)
		{
			Socket socket = null;
			try {
				socket = server.accept();
				// Nuova connessione
				System.out.println("New Connection on port : " + socket.getLocalPort());
			} catch (IOException e) {
				System.out.println("Error while enstablishing connection");
				e.printStackTrace();
				continue;
			}

			// Crea stream di in/out
			ObjectOutputStream outSocket;
			ObjectInputStream inSocket;
			try {
				outSocket = new ObjectOutputStream(socket.getOutputStream());
				inSocket = new ObjectInputStream(socket.getInputStream());

			} catch (IOException e) {
				System.out.println("Unable to get socket streams");
				e.printStackTrace();
				continue;
			}

			// Adesso la connessione è correttamente stabilita
			try {
				SecretKey key = AESDHKeyAgreement(outSocket, inSocket);

				String cleartext = receiveAESCryptedString(key, inSocket);
				System.out.println("Server: decrypted text: "+cleartext);

				int times = Integer.parseInt(cleartext);
				for(int i = 0; i<times; i++) {
					String SysLoadAvg = String.valueOf(ManagementFactory.
					getOperatingSystemMXBean().getSystemLoadAverage());
					sendAESCryptedString(SysLoadAvg, key, outSocket);
					Thread.sleep(2000);
				}
				sendAESCryptedString("stop", key, outSocket);

			} catch (Exception e) {
				System.out.println("Server: ERROR");
				e.printStackTrace();
				continue;
			}
		}
	}

	public SecretKey AESDHKeyAgreement(ObjectOutputStream outSocket, ObjectInputStream inSocket) throws Exception {

		/*
		 * Il Server riceve la chiave pubblica DH del Client in forma codificata.
		 * Istanzia un oggetto PublicKey dai dati che ha ricevuto
		 */
		byte[] clientPubKeyEnc = (byte[]) inSocket.readObject();
		KeyFactory serverKeyFac = KeyFactory.getInstance("DH");
		X509EncodedKeySpec x509KeySpec = new X509EncodedKeySpec(clientPubKeyEnc);
		PublicKey clientPubKey = serverKeyFac.generatePublic(x509KeySpec);

		/*
		 * Il Server preleva i parametri di DH associati alla chiave pubblica del Client.
		 * PK = (g,p,A)
		 * p: numero primo
		 * g: generatore del gruppo moltiplicativo degli interi modulo p
		 * A: (g^a) mod p
		 * Gli serviranno per generare la sua coppia di chiavi DH
		 */
		DHParameterSpec dhParamSpec = ((DHPublicKey) clientPubKey).getParams();

		/* Il Server genera la sua coppia di chiavi DH
		 * SK = b: numero casuale
		 * PK = B: (g^b) mod p
		 */
		System.out.println("Server: Generate DH keypair ...");
		KeyPairGenerator serverKpairGen = KeyPairGenerator.getInstance("DH");
		serverKpairGen.initialize(dhParamSpec);
		KeyPair serverKpair = serverKpairGen.generateKeyPair();

		// Il Server codifica la sua chiave pubblica e la invia al Client
		byte[] serverPubKeyEnc = serverKpair.getPublic().getEncoded();
		outSocket.writeObject(serverPubKeyEnc);

		/* Il Server crea ed inizializza un oggetto KeyAgreement di DH che,
		 * grazie alla SK del Server e alla PK del Client,
		 * può calcolare la chiave concordata K
		 * K = (clientPK^b) mod p = (A^b) mod p =(g^ab) mod p
		 */
		System.out.println("Server: Initialization ...");
		KeyAgreement serverKeyAgree = KeyAgreement.getInstance("DH");
		serverKeyAgree.init(serverKpair.getPrivate());
		System.out.println("Server: calculating agreed KEY ...");
		serverKeyAgree.doPhase(clientPubKey, true);

		/*
		 * A questo punto sia il Client che il Server hanno completato il protocollo di DH.
		 * Hanno ottenuto entrambi la chiave concordata K
		 */
		return serverKeyAgree.generateSecret("AES");
	}

	private void sendAESCryptedString(String cleartext, SecretKey key, ObjectOutputStream outSocket) throws Exception {
		Cipher cipher = Cipher.getInstance("AES/CBC/PKCS5Padding");
		cipher.init(Cipher.ENCRYPT_MODE, key);

		byte[] encodedParams = cipher.getParameters().getEncoded();
		outSocket.writeObject(encodedParams);

		byte[] ciphertext = cipher.doFinal(cleartext.getBytes());
		outSocket.writeObject(ciphertext);
	}

	private String receiveAESCryptedString(SecretKey key, ObjectInputStream inSocket) throws Exception {
		byte[] encodedParams = (byte[]) inSocket.readObject();
		AlgorithmParameters params = AlgorithmParameters.getInstance("AES");
		params.init(encodedParams);

		Cipher cipher = Cipher.getInstance("AES/CBC/PKCS5Padding");
		cipher.init(Cipher.DECRYPT_MODE, key, params);
		byte[] ciphertext = (byte[]) inSocket.readObject();
		byte[] recovered = cipher.doFinal(ciphertext);
		return new String(recovered);
	}
}
\end{lstlisting}

\end{document}
